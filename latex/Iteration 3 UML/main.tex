\documentclass[a4paper,12pt]{article}
\usepackage[utf8]{inputenc}
\usepackage[T1]{fontenc}
\usepackage[frenchb]{babel} % If you write in French
\usepackage{a4wide}
\usepackage{graphicx}
\graphicspath{{images/}}
\usepackage{subfig}
\usepackage{tikz}
\usetikzlibrary{shapes,arrows}
\usepackage{pgfplots}
\pgfplotsset{compat=newest}
\pgfplotsset{plot coordinates/math parser=false}
\newlength\figureheight
\newlength\figurewidth
\pgfkeys{/pgf/number format/.cd,
	set decimal separator={,\!},
	1000 sep={\,},
}

\usepackage{ifthen}
\usepackage{ifpdf}
\ifpdf
\usepackage[pdftex]{hyperref}
\else
\usepackage{hyperref}
\fi
\usepackage{color}
\hypersetup{%
	colorlinks=true,
	linkcolor=black,
	citecolor=black,
	urlcolor=black}

\renewcommand{\baselinestretch}{1.05}
\usepackage{fancyhdr}
\pagestyle{fancy}
\fancyfoot{}
\fancyhead[LE,RO]{\bfseries\thepage}
\fancyhead[RE]{\bfseries\nouppercase{\leftmark}}
\fancyhead[LO]{\bfseries\nouppercase{\rightmark}}
\setlength{\headheight}{20pt}

\let\headruleORIG\headrule
\renewcommand{\headrule}{\color{black} \headruleORIG}
\renewcommand{\headrulewidth}{1.0pt}
\usepackage{colortbl}
\arrayrulecolor{black}

\fancypagestyle{plain}{
	\fancyhead{}
	\fancyfoot[C]{\thepage}
	\renewcommand{\headrulewidth}{0pt}
}

\makeatletter
\def\@textbottom{\vskip \z@ \@plus 1pt}
\let\@texttop\relax
\makeatother

\makeatletter
\def\cleardoublepage{\clearpage\if@twoside \ifodd\c@page\else%
	\hbox{}%
	\thispagestyle{empty}%
	\newpage
	\if@twocolumn\hbox{}\fi\fi\newpage\fi}
\makeatother

\usepackage{amsthm}
\usepackage{amssymb,amsmath,bbm}
\usepackage{array}
\usepackage{bm}
\usepackage{multirow}
\usepackage[footnote]{acronym}

\begin{document}
	\begin{titlepage}
		\begin{center}
			
			\includegraphics[width=0.6\textwidth]{img/logop8-2}\\[1cm]
			
			{\large Département informatique, Licence 3\up{ème} année}\\[0.5cm]
			
			{\large Réalisation d'applications}\\[0.5cm]
			
			% Title
			\rule{\linewidth}{0.5mm} \\[0.4cm]
			{ \huge \bfseries iteration 3: UML Gestion de stock \\[0.4cm] }
			\rule{\linewidth}{0.5mm} \\[1.5cm]
			
			 
			% Author and supervisor
			\noindent
			\begin{minipage}{0.4\textwidth}
				\begin{flushleft} \large
					\emph{Auteur :}\\
					 \textsc{QUERIC Yann }\\
					 \textsc{LEGOUEIX Nicolas}\\
					 \textsc{COTREZ Leo}\\
					 \textsc{ORNIACKI Thomas}\\
					
				\end{flushleft}
			\end{minipage}%
			\begin{minipage}{0.4\textwidth}
				\begin{flushright} \large
					\emph{Enseignant :} \\
					M \textsc{P.Lefebvre}\\
				\end{flushright}
			\end{minipage}
			
			\vfill
			
			% Bottom of the page
			{\large \today }
			
		\end{center}
	\end{titlepage}
	\tableofcontents
	\newpage
	\section{Préliminaire}
	    \subsection{Paquetage}
	    	\includegraphics[width=0.9\textwidth]{img/classDiagram.png}
	    \newpage
	    \subsection{MVC}
	    \includegraphics[width=0.9\textwidth]{img/page1.jpg}\\
	    \subsubsection{Caisse}
	    \begin{itemize}
	        \item Model
    	        \begin{itemize}
        	        \item formate les données pour les données du controleur pour la vue
    	        \end{itemize}
	        \item View
	            \begin{itemize}
        	        \item (voir page1.jpg)
        	        \item aperçu des acticles scrollable
        	        \item boutons
    	        \end{itemize}
	        \item Control
	            \begin{itemize}
    	            \item pavé numerique: saisir le code produit
                    \item 1 boutton total: 2 fonctionnalités
    		        \item 1 appuie couvre la caisse et imprime un ticket de la transaction couvre la caisse
    		    	\item 2 appuie couvre la caisse et imprime le ticket de l'ensemble des transaction depuis le dernier solde ainsi que le total
                    \item 1 entrée: validé la saisi du code produit
                    \item 1 retour: initie la procedure de retour en attente du code produit puis ouverture de la caisse pour le rembourcement puis a la fermeture du tiroir incrementation de la quantité en stock
                    \item 1 lecteur laser: scan le code barre de l'acticle
                    \item 1 bouttons + - dans la collone quantité pour augmenter ou diminuer la quantité d'un produit
    	        \end{itemize}
	    \end{itemize}
	    \includegraphics[width=0.9\textwidth]{img/page6.png}
	    \subsubsection{Affichage des produits en stock}
	    \begin{itemize}
	        \item Model
    	        \begin{itemize}
        	        \item formate les données pour les données du controleur pour la vue
    	        \end{itemize}
	        \item View
	            \begin{itemize}
        	        \item (voir page6.jpg)
        	        \item affichage liste des produit en stock
        	        \item boutons
    	        \end{itemize}
	        \item Control
	            \begin{itemize}
                    \item 1 bouton pour acceder a la liste des fournisseurs (passage au MVC liste fourniseur)
                    \item 1 bouton pour acceder a la liste stock (passage au MVC liste stock)
                    \item 1 bouton ajouter article (passage au MVC ajouter un article)
                    \item 1 bouton ajouter fournisseurs (passage au MVC ajouter un fournisseur)
                    \item 1 bouton impression des produits en stock (demande de confirmation)
                    \item 1 bouton pour accéder au acticle sous leurs seuil de rupture (passage au MVC liste des article sous le seuil)
                    \item cliquer sur un produit pour acceder au detail (passage au MVC detail de produit)
                    \item 1 bouton pour acceder au commande en cours (passage au MVC liste en cours de commande)
    	        \end{itemize}
	    \end{itemize}
	    \includegraphics[width=0.9\textwidth]{img/page8.png}
	    \subsubsection{detail d'un produit }
	    \begin{itemize}
	        \item Model
    	        \begin{itemize}
        	        \item formate les données pour les données du controleur pour la vue
    	        \end{itemize}
	        \item View
	            \begin{itemize}
        	        \item (voir page8.jpg)
        	        \item affichage liste des details du produit (prix, fournisseur,...)
        	        \item boutons
    	        \end{itemize}
	        \item Control
	            \begin{itemize}
                    \item 1 bouton pour acceder a la liste des fournisseurs (passage au MVC liste fourniseur)
                    \item 1 bouton pour acceder a la liste stock (passage au MVC liste stock)
                    \item 1 bouton ajouter article (passage au MVC ajouter un article)
                    \item 1 bouton ajouter fournisseurs (passage au MVC ajouter un fournisseur)
                    \item 1 bouton impression des details d'un produit (demande de confirmation)
                    \item 1 bouton edition des details du produit (demande de confirmation)
                    \item 1 bouton suppression d'un produit (demande de confirmation)
                    \item 1 bouton de reaprovisionement d'un produit (passage au MVC de demande de reaprovisionnement)
    	        \end{itemize}
	    \end{itemize}
	    \includegraphics[width=0.9\textwidth]{img/page2.jpg}
	    \subsubsection{ajout d'un produit }
	    \begin{itemize}
	        \item Model
    	        \begin{itemize}
        	        \item formate les données pour les données du controleur pour la vue
    	        \end{itemize}
	        \item View
	            \begin{itemize}
        	        \item (voir page2.jpg)
        	        \item formulaire ajout de produit
        	        \item boutons
    	        \end{itemize}
	        \item Control
	            \begin{itemize}
                    \item 1 bouton pour acceder a la liste des fournisseurs (passage au MVC liste fourniseur)
                    \item 1 bouton pour acceder a la liste stock (passage au MVC liste stock)
                    \item 1 bouton ajouter article (passage au MVC ajouter un article)
                    \item 1 bouton ajouter fournisseurs (passage au MVC ajouter un fournisseur)
                    \item 1 bouton de valider l'ajout de produit 
    	        \end{itemize}
	    \end{itemize}
	    \includegraphics[width=0.9\textwidth]{img/page4}
	    \subsubsection{liste des fournisseurs }
	    \begin{itemize}
	        \item Model
    	        \begin{itemize}
        	        \item formate les données pour les données du controleur pour la vue
    	        \end{itemize}
	        \item View
	            \begin{itemize}
        	        \item (voir page4.jpg)
        	        \item liste des fournisseurs
        	        \item barre de recherche
        	        \item bouton cliquables
    	        \end{itemize}
	        \item Control
	            \begin{itemize}
                    \item 1 bouton pour acceder a la liste des fournisseurs (passage au MVC liste fourniseur)
                    \item 1 bouton pour acceder a la liste stock (passage au MVC liste stock)
                    \item 1 bouton ajouter article (passage au MVC ajouter un article)
                    \item 1 bouton ajouter fournisseurs (passage au MVC ajouter un fournisseur)
                    \item 1 barre de text pour taper la recherche
                    \item 1 bouton pour valider la recherche
                    \item cliquer sur un fournisseur pour acceder au detail (passage au MVC detail fournisseur)
    	        \end{itemize}
	    \end{itemize}
	    \includegraphics[width=0.9\textwidth]{img/page7.jpg}
	    \subsubsection{Affichage des details fournisseurs }
	    \begin{itemize}
	        \item Model
    	        \begin{itemize}
        	        \item formate les données pour les données du controleur pour la vue
    	        \end{itemize}
	        \item View
	            \begin{itemize}
        	        \item (voir page7.jpg)
        	        \item affichage des details fournisseursb
        	        \item bouton
    	        \end{itemize}
	        \item Control
	            \begin{itemize}
                    \item 1 bouton pour acceder a la liste des fournisseurs (passage au MVC liste fourniseur)
                    \item 1 bouton pour acceder a la liste stock (passage au MVC liste stock)
                    \item 1 bouton ajouter article (passage au MVC ajouter un article)
                    \item 1 bouton ajouter fournisseurs (passage au MVC ajouter un fournisseur)
                    \item 1 bouton impression des details fournisseur(demande de confirmation)
                    \item 1 bouton suppression fournisseur (passage au MVC detail de fournisseur)
    	        \end{itemize}
	    \end{itemize}
	    \includegraphics[width=0.9\textwidth]{img/page3.jpg}
	    \subsubsection{ajout d'un fournisseur }
	    \begin{itemize}
	        \item Model
    	        \begin{itemize}
        	        \item formate les données pour les données du controleur pour la vue
    	        \end{itemize}
	        \item View
	            \begin{itemize}
        	        \item (voir page3.jpg)
        	        \item formulaire ajout de fournisseur
        	        \item bouton
    	        \end{itemize}
	        \item Control
	            \begin{itemize}
                    \item 1 bouton pour acceder a la liste des fournisseurs (passage au MVC liste fourniseur)
                    \item 1 bouton pour acceder a la liste stock (passage au MVC liste stock)
                    \item 1 bouton ajouter article (passage au MVC ajouter un article)
                    \item 1 bouton ajouter fournisseurs (passage au MVC ajouter un fournisseur)
                    \item 1 bouton de valider l'ajout de fournisseur (passage au MVC detail de fournisseur)
    	        \end{itemize}
	    \end{itemize}
	    \includegraphics[width=0.9\textwidth]{img/page6.png}
	    \subsubsection{Affichage des produits sous leurs seuil }
	    \begin{itemize}
	        \item Model
    	        \begin{itemize}
        	        \item formate les données pour les données du controleur pour la vue
    	        \end{itemize}
	        \item View
	            \begin{itemize}
        	        \item (voir page6.jpg)
        	        \item affichage liste des produit sous leurs seuil
        	        \item bouton
    	        \end{itemize}
	        \item Control
	            \begin{itemize}
                    \item 1 bouton pour acceder a la liste des fournisseurs (passage au MVC liste fourniseur)
                    \item 1 bouton pour acceder a la liste stock (passage au MVC liste stock)
                    \item 1 bouton ajouter article (passage au MVC ajouter un article)
                    \item 1 bouton ajouter fournisseurs (passage au MVC ajouter un fournisseur)
                    \item 1 bouton impression des produits sous leurs seuil (demande de confirmation)
                    \item cliquer sur un produit pour voir les details (passage au MVC detail de produit)
                    \item 1 bouton pour acceder au commande en cours (passage au MVC liste en cours de commande)
    	        \end{itemize}
	    \end{itemize}
	    \includegraphics[width=0.9\textwidth]{img/page9.png}
	    \subsubsection{Affichage liste commande }
	    \begin{itemize}
	        \item Model
    	        \begin{itemize}
        	        \item formate les données pour les données du controleur pour la vue
    	        \end{itemize}
	        \item View
	            \begin{itemize}
        	        \item (voir page9.jpg)
        	        \item affichage liste des commande
        	        \item bouton
    	        \end{itemize}
	        \item Control
	            \begin{itemize}
                    \item 1 bouton pour acceder a la liste des fournisseurs (passage au MVC liste fourniseur)
                    \item 1 bouton pour acceder a la liste stock (passage au MVC liste stock)
                    \item 1 bouton ajouter article (passage au MVC ajouter un article)
                    \item 1 bouton ajouter fournisseurs (passage au MVC ajouter un fournisseur)
                    \item 1 bouton impression des produits de liste des commande (demande de confirmation)
    	        \end{itemize}
	    \end{itemize}
	    \includegraphics[width=0.9\textwidth]{img/page10.png}
	    \subsubsection{creation d'un commande }
	    \begin{itemize}
	        \item Model
    	        \begin{itemize}
        	        \item formate les données pour les données du controleur pour la vue
    	        \end{itemize}
	        \item View
	            \begin{itemize}
        	        \item (voir page10.jpg)
        	        \item formulaire creation de commande (avec choix du fournisseurs, et quantité)
        	        \item bouton
    	        \end{itemize}
	        \item Control
	            \begin{itemize}
                    \item 1 bouton pour acceder a la liste des fournisseurs (passage au MVC liste fourniseur)
                    \item 1 bouton pour acceder a la liste stock (passage au MVC liste stock)
                    \item 1 bouton ajouter article (passage au MVC ajouter un article)
                    \item 1 bouton ajouter fournisseurs (passage au MVC ajouter un fournisseur)
                    \item 1 bouton de validation de la commande (demande de confirmation)
                    \item //impression du bon de commande apres validation
    	        \end{itemize}
	    \end{itemize}
\end{document}
